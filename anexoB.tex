\chapter{\normalsize Programa base para la HMI en Python}
\vspace{0cm}
Autores: Carlos Xavier Rosero, Manel Velasco Garc'ia. Nombre y extensi'on de archivo: \textbf{\textit{testDimitris.py}}.
{\tiny
\begin{lstlisting}[language=python]
#!/usr/bin/env python
# -*- coding: utf-8 -*-
"""
@authors: carlos xavier rosero
          manel velasco garc�a

best performance tested with python 2.7.3, GCC 2.6.3, pyserial 2.5
"""
from __future__ import division

import serial, struct
from collections import deque
import time

import matplotlib.pyplot as plt
import matplotlib.animation as animation

class command:
    startCx = 0xf0
    stopCx = 0xf1


#use this part only with old versions of python and libraries
#tested specifically with python 2.6.5, GCC 4.4.3, serial 1.3.5
class adapt(serial.Serial):
    def write(self, data):
        super(self.__class__, self).write(str(data))

class serialCx:

    def __init__(self, usbPort, frameLen, maxLen, EOL, bauds):

        self.ay1 = deque([0.0]*maxLen)
        self.ay2 = deque([0.0]*maxLen)

        self.maxLen = maxLen

        self.eol = EOL #end of line
        self.lenEol = len(self.eol) #length of the end of line

        self.frameLen = frameLen - self.lenEol

        self.float0 = float(0)
        self.float1 = float(0)

        # open serial port
        self.ser = adapt(port=usbPort, baudrate=bauds, timeout=1, parity=serial.PARITY_NONE,
                stopbits=serial.STOPBITS_ONE)

        self.ser.open
        self.ser.isOpen

        time.sleep(2) #waits until arduino gets up

        inputS = [] #initializes buffer
        while self.ser.inWaiting() > 0: #neglects the trash code received
            inputS.append(self.ser.read(1)) #appends a new value into the buffer

        
        self.float0ToTx = float(9.5)  #initial values of the floating numbers to be sent
        self.float1ToTx = float(-9.5)

        value0 = struct.pack('%sf' % 1, self.float0ToTx) #splits the float value into 4 strings
        value1 = struct.pack('%sf' % 1, self.float1ToTx) #splits the float value into 4 strings

        #this buffer sends the command to start tx from arduino and both floating registers
        buffer = bytearray([command.startCx, ord(value0[0]), ord(value0[1]), ord(value0[2]), ord(value0[3]),
                                             ord(value1[0]), ord(value1[1]), ord(value1[2]), ord(value1[3])])

        self.ser.write(buffer) #sends the command to start the reception


  # add to buffer
    def addToBuf(self, buf, val):
        if len(buf) < self.maxLen:
            buf.append(val)
        else:
            buf.pop()
            buf.appendleft(val)

    # update plot
    def update(self, frameNum, a1, a2):
        line = bytearray()
        try:
            while True:
                c = self.ser.read(1)
                if c:
                    line.append(c)
                    if line[-self.lenEol:] == self.eol: #verifies if EOF has been received
                        line = line[0:-self.lenEol] #removes EOF
                        break
                else:
                    break

            if len(line) == self.frameLen:

                self.float0 = struct.unpack('f', line[0:4])
                self.float0 = self.float0[0] #takes out the only one element from the tuple

                self.float1 = struct.unpack('f', line[4:8])
                self.float1 = self.float1[0] #takes out the only one element from the tuple

                print self.float0,
                print self.float1

                # add data to buffer
                self.addToBuf(self.ay1, self.float0)
                self.addToBuf(self.ay2, self.float1)

                a1.set_data(range(self.maxLen), self.ay1)
                a2.set_data(range(self.maxLen), self.ay2)

            else:
                print 'Incorrect frame length:',
                print len(line)
                self.ser.flushInput() #gets empty the serial port buffer

        except KeyboardInterrupt:
            print('Exiting...')

    # clean up
    def close(self):
        # close serial

        buffer = bytearray([command.stopCx, 0x01, 0x01, 0x01, 0x01, 0x01, 0x01,
                            0x01, 0x01]) #the (0x01) is used only to fill the buffer
                                         #in order to accomplish the standard length

        self.ser.write(buffer) #sends the command to start the reception

        self.ser.flushInput()
        self.ser.close()

def plotting(givenData):

    fig = plt.figure()

    ax1 = plt.subplot(211)
    ax1.grid(True)
    a1, = ax1.plot([0], [0], color=(0, 1, 0))
    plt.xlim(0, 200)
    plt.ylim(-10, 10)
    plt.ylabel('Float 0')

    ax2 = plt.subplot(212)
    ax2.grid(True)
    a2, = ax2.plot([0], [0], color=(0, 1, 0))
    plt.xlim(0, 200)
    plt.ylim(-10, 10)
    plt.ylabel('Float 1')

    anim = animation.FuncAnimation(fig, givenData.update, fargs=(a1, a2), interval=50)

    plt.show() # show plot

#######################################
######here starts the main program#####
#######################################
if __name__ == "__main__":

    testCarlos = serialCx('/dev/ttyACM9', 12, 200, 'cXrC', 115200)
    plotting(testCarlos)
    serialCx.close(testCarlos)
\end{lstlisting}

}


