
\prefacesection{\normalsize Resumen }

Un p'endulo invertido es un dispositivo que consiste en una barra cil'indrica con 
libertad de oscilar alrededor de un pivot fijo, esta montado sobre un carro que sigue  una  trayectoria  horizontal. El objetivo es mantener el p'endulo perpendicular a la trayectoria del carro ante la presencia de perturbaciones en el sistema; el sistema corrige la posici'on angular del p'endulo desplazando el carro con un sistema de banda-motor, seg'un una acci'on de control calculada.

El controlador implementado para el sistema fue desarrollado mediante la t'ecnica de espacio de estados a partir del modelo matem'atico y simulado en Matlab obteniendo sus ganancias, las mismas que sirven para modificar el comportamiento del sistema. El sistema de control est'a divido en dos, una parte de monitoreo de datos o HMI y otra de adquisici'on de datos y control. 

El sistema de adquisici'on de datos est'a montado en 4 nodos controlados por Arduinos dentro de una red CAN; el nodo 0 o nodo central tiene la tarea de recibir datos desde la red, procesarlos y tomar una acci'on de control; tambi'en tiene la importante tarea de servir de enlace entre la HMI y el resto de la red. El nodo 1, adquiere el dato de posici'on del carro y la envia al nodo 2; 'este a su vez toma el dato del 'angulo del p'endulo y lo envia al nodo 0. El nodo 4 en conjunto con un puente H toma la acci'on de control enviada desde el nodo 0 y la convierte en voltaje el cual controla el motor enlazado al carro.

El sistema de monitoreo o HMI desarrollada en PyCharm muestra los datos provenientes de la red, como la posici'on y velocidad del carro y del p'endulo. Desde la HMI se controla la red de comunicaciones y se puede enviar datos de ganancias para el controlador en el nodo central.

